\documentclass[a4paper,14pt, unknownkeysallowed]{extreport}
\usepackage{placeins}
\include{preamble.inc}

\begin{document}

\include{00-title}

\section*{Задание 1}
Создать базу знаний «Предки», позволяющую наиболее эффективным способом
(за меньшее количество шагов, что обеспечивается меньшим количеством предложений БЗ - правил), и используя разные варианты (примеры) простого вопроса определить:
\begin{enumerate}
	\item по имени субъекта определить всех его бабушек,
	\item по имени субъекта определить всех его дедушек,
	\item по имени субъекта определить всех его бабушек и дедушек,
	\item по имени субъекта определить его бабушку по материнской линии,
	\item по имени субъекта определить его бабушку и дедушку по материнской линии.
\end{enumerate}

\begin{lstlisting}
parent(olga, oleg, f).
parent(ivan, oleg, m).
parent(oleg, yaroslav, m).
parent(inga, yaroslav, f).
parent(oleg, kate, m).
parent(inga, kate, f).
parent(nikita, georg, m).
parent(sveta, georg, f).
parent(yaroslav, stepan, m).
parent(irina, stepan, f).
parent(georg, sergey, m).
parent(kate, sergey, f).
parent(georg, alex, m).
parent(kate, alex, f).
parent(georg, nika, m).
parent(kate, nika, f).
grandparent(X, Y)   :- parent(X, Z, _), parent(Z, Y, _).
grandmother(X, Y)   :- parent(X, Z, f), parent(Z, Y, _).
grandfather(X, Y)   :- parent(X, Z, m), parent(Z, Y, _).
maternal_grandmother(X, Y) :- parent(X, Z, f), parent(Z, Y, f).
maternal_grandparent(X, Y) :- parent(X, Z, _), parent(Z, Y, f).	
\end{lstlisting}

\section*{Задание 2}

Дополнить базу знаний правилами, позволяющими найти
\begin{enumerate}
	\item Максимум из двух чисел
	\begin{enumerate}
		\item без использования отсечения,
		\item с использованием отсечения;
	\end{enumerate}
	\item Максимум из трех чисел
	\begin{enumerate}
		\item без использования отсечения,
		\item с использованием отсечения;
	\end{enumerate}
\end{enumerate}

\begin{lstlisting}
max2(N1, N2, N2)    :- N2 >= N1.
max2(N1, N2, N1)    :- N1 >= N2.

max3(N1, N2, N3, N1)    :- N1 >= N2, N1 >= N3.
max3(N1, N2, N3, N2)    :- N2 >= N1, N2 >= N3.
max3(N1, N2, N3, N3)    :- N3 >= N1, N3 >= N2.

max2clip(N1, N2, N2)    :- N2 >= N1, !.
max2clip(N1, _, N1).

max3clip(N1, N2, N3, N1)   :- N1 >= N2, N1 >= N3, !.
max3clip(_, N2, N3, N2)    :- N2 >= N3, !.
max3clip(_, _, N3, N3).
\end{lstlisting}

\begin{table}[ht!]
	\centering
	\label{decisions}
	\begin{tabular}{|p{1.1cm}|p{5cm}|p{6cm}|p{5cm}|}
		\hline
		\textbf{Шаг} & \textbf{Состояние резольвенты} & \textbf{Для каких термов запускается алгоритм унификации и результат подстановки} & \textbf{Дальнейшие действия: прямой ход или откат}\\
		\hline
		1 & max3(3, 2, 1, Asw) & max3(3, 2, 1, Asw) = max2(N1, N2, N2) Подстановка: нет & Откат \\
		\hline
		... & & & \\
		\hline
		3 & 3 >= 2, 3 >= 1 & max3(3, 2, 1, Asw) = max3(N1, N2, N3, N1) Подстановка:\{N1=3; N2=2; N3=1; Asw=3\} & Прямой ход \\
		\hline
		4 & 3 >= 1 & 3 >= 2 Верно. Подстановка:\{N1=3; N2=2; N3=1; Asw=3\} & Прямой ход \\
		\hline
		5 & Пусто & 3 >=1 Верно. Подстановка:\{N1=3; N2=2; N3=1; Asw=3\} & Конкретизация Asw. Откат к пункту 3. Дальнейшее продолжение перебора. \\
		\hline
	\end{tabular}
\end{table}

\begin{table}[ht!]
	\centering
	\label{decisions}
	\begin{tabular}{|p{1.1cm}|p{5cm}|p{6cm}|p{5cm}|}
		\hline
		\textbf{Шаг} & \textbf{Состояние резольвенты} & \textbf{Для каких термов запускается алгоритм унификации и результат подстановки} & \textbf{Дальнейшие действия: прямой ход или откат}\\
		\hline
		1 & max3clip(2, 3, 1, Asw) & max3clip(2, 3, 1, Asw) = max2(N1, N2, N2) Подстановка: нет & Откат \\
		\hline
		... & & & \\
		\hline
		8 & 2 >= 3, 2 >= 1 & max3clip(2, 3, 1, Asw) = max3clip(N1, N2, N3, N1) Подстановка:\{N1=2; N2=3; N3=1; Asw=2\} & Прямой ход \\
		\hline
		9 & 2 >= 1 & 2 >= 3 Неверно. & Откат к пункту 8 \\
		\hline
		10 & 3 >= 1 & max3clip(2, 3, 1, Asw) = max3clip(\_, N2, N3, N2) Подстановка: \{N2=3; N3=1; Asw=3\} & Прямой ход \\
		\hline
		11 & Пусто & 3 >= 1 Верно. Подстановка: \{N2=3; N3=1; Asw=3\} & Конкретизация Asw. Оператор отсечения. Откат к пункту 8. Завершение работы.\\
		\hline
	\end{tabular}
\end{table}

\end{document}