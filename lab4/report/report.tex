\documentclass[a4paper,14pt, unknownkeysallowed]{extreport}
\usepackage{placeins}
\include{preamble.inc}

\begin{document}

\include{00-title}

\section*{Задание 1}

\textbf{Чем принципиально отличаются функции cons, list, append?}

\begin{itemize}
	\item cons --- принимает 2 аргумента. Если второй аргумент является списком, создает 1 списковую ячейку, значением которой является первый аргумент, а список -- ссылкой на второй аргумент (список). Если второй аргумент является атомом, создает точечную пару.
	\item list --- принимает переменное количество аргументов. Создает столько списковых ячеек, сколько принято аргументов. Значением каждой списковой ячейки является значение принятых аргументов.
	\item append --- принимает переменное количество аргументов. все аргументы, кроме последнего, должны быть списками. Возвращает новый список, содержащий конкатенацию копий. Списки остаются без изменений, копируется структура каждого списка, кроме последнего: он не копируется, а становится cdr последней точечной пары или возвращается напрямую, если нет предшествующих непустых списков.
\end{itemize}

\textbf{Пусть (setf lst1 '( a b c))
	(setf lst2 '( d e)).\\
	Каковы результаты вычисления следующих выражений?}

\begin{itemize}
	\item (cons lstl lst2) --- ((A B C) D E)
	\item (list lst1 lst2) --- ((A B C) (D E))
	\item (append lst1 lst2) --- (A B C D E)
\end{itemize}

\newpage

\section*{Задание 2}

\begin{itemize}
	\item (reverse '(a b c)) 		--- (C B A) 
	\item (reverse '(a b (c (d))))	--- ((C (D)) B A) 
	\item (reverse '(a)) 			--- (A) 
	\item (reverse ()) 				--- NIL
	\item (reverse '((a b c))) 		--- ((A B C)) 
	\item (last '(a b c)) 			--- (C) 
	\item (last '(a)) 				--- (A) 
	\item (last '((a b c))) 		--- ((A B C)) 
	\item (last '(a b (c))) 		--- ((C)) 
	\item (last ()) 				--- NIL
\end{itemize}

\section*{Задание 3}

\textbf{Написать, по крайней мере, два варианта функции, которая возвращает последний элемент своего списка-аргумента.}

\begin{lstlisting}
(defun last_el (lst)
	(car (last lst))
)
\end{lstlisting}

\begin{lstlisting}
(defun last_el (lst)
	(car (reverse lst))
)
\end{lstlisting}

\newpage

\section*{Задание 4}

\textbf{Написать, по крайней мере, два варианта функции, которая возвращает свой список аргумент без последнего элемента.}

\begin{lstlisting}
(defun without_last_el (lst)
	(
		reverse (cdr (reverse lst))
	)
)
\end{lstlisting}

\begin{lstlisting}
(defun without_last_el (lst)
	(
		cond 
		(
			(cdr lst) 
			(cons (car lst) (without_last_el (cdr lst)))
		)
		(
			T
			Nil
		)
	)
)
\end{lstlisting}

\section*{Задание 5}

\textbf{Напишите функцию swap-first-last, которая переставляет в списке-аргументе первый и последний элементы.}

\begin{lstlisting}
(defun swap_first_last (lst)
	(
		append 
		(last lst)
		(reverse (cdr (reverse (cdr lst))))
		(cons (car lst) Nil)
	)
)
\end{lstlisting}

\section*{Задание 6}

\textbf{Написать простой вариант игры в кости, в котором бросаются две правильные кости. Если сумма выпавших очков равна 7 или 11 --- выигрыш, если выпало (1,1) или (6,6) --- игрок имеет право снова бросить кости, во всех остальных случаях ход переходит ко второму игроку, но запоминается сумма выпавших очков. Если второй игрок не выигрывает абсолютно, то выигрывает тот игрок, у которого больше очков. Результат игры и значения выпавших костей выводить на экран с помощью функции print.}

\begin{lstlisting}
(defvar rs (make-random-state T))
(defvar thr1)
(defvar thr2)

(defun bones_throw ()
	(
		list
		(+ (random 6 rs) 1)
		(+ (random 6 rs) 1)
	)
)

(defun win_check (lst)
	(
		or
		(   
			= 
			(+ (car lst) (cadr lst))
			7
		)
		(   
			= 
			(+ (car lst) (cadr lst))
			11
		)
	)
)


(defun pass_check (lst)
	(
		or
		(   
			and 
			(= (car lst) 1)
			(= (cadr lst) 1)
		)
		(   
			and 
			(= (car lst) 6)
			(= (cadr lst) 6)
		)
	)
)

(defun gameplay2 ()
	(princ "Second player throw: ") 
	(setq thr2 (bones_throw))
	(princ thr2)
	(terpri)
	(
		cond 
		(
			(win_check thr2)
			(princ "Second player wins!") 
		)
		(
			(pass_check thr2)
			(gameplay2)
		)
		(
			T
			(
				cond
				(
					(
						>
						(+ (car thr1) (cadr thr1))
						(+ (car thr2) (cadr thr2))
					)
					(princ "First player wins!") 
				)
				(
					(
						<
						(+ (car thr1) (cadr thr1))
						(+ (car thr2) (cadr thr2))
					)
					(princ "Second player wins!") 
				)
				(
					T
					(princ "Draw in the game!")
				)
			)
		)
	)
)
(defun gameplay1 ()
	(princ "First player throw: ") 
	(setq thr1 (bones_throw))
	(princ thr1)
	(terpri)
	(
		cond 
		(
			(win_check thr1)
			(princ "First player wins!") 
		)
		(
			(pass_check thr1)
			(gameplay1)
		)
		(
			T
			(gameplay2)
		)
	)
	(terpri)
)
(gameplay1)
\end{lstlisting}

\section*{Задание 7}

\textbf{Написать функцию, которая по своему списку-аргументу lst определяет является ли он палиндромом (то есть равны ли lst и (reverse lst)).}

\begin{lstlisting}
(defun palindrom_check (lst)
	(defun st_check (ls revls n)
		(
			cond
			(
				(> n 1)
				(
					and
					(= (car ls) (car revls))
					(st_check (cdr ls) (cdr revls) (- n 2))
				)
			)
			(
				T 
				T
			)
		)
	)
	(st_check lst (reverse lst) (length lst))
)
\end{lstlisting}

\section*{Задание 8}

\textbf{Напишите свои необходимые функции, которые обрабатывают таблицу из 4-х точечных пар: (страна . столица), и возвращают по стране --- столицу, а по столице --- страну.}

\newpage

\begin{lstlisting}
(defun countries_capitals (lst name)
	(
		cond 
		(
			(
				equal
				(caar lst)
				name
			)
			(cdar lst)
		)
		(
			(
				equal
				(cdar lst)
				name
			)
			(caar lst)
		)
		(
			(cdr lst)
			(countries_capitals (cdr lst) name)
		)
	)
)
\end{lstlisting}

\section*{Задание 9}

\textbf{Напишите функцию, которая умножает на заданное число-аргумент первый числовой элемент списка из заданного 3-х элементного списка- аргумента, когда
	a) все элементы списка --- числа,
	6) элементы списка -- любые объекты}

\begin{lstlisting}
(defun mult_el_a (n lst)
	(
		cond 
		(
			(
				and
				(
					and
					(numberp (car lst))
					(
						and
						(numberp (cadr lst))
						(numberp (caddr lst))
					)
				)
				(numberp n)
			)
			(* (car lst) n)
		)
		(
			T 
			Nil
		)
	)
)

(defun mult_el_b (n lst)
	(
		cond 
		(
			(
				and
				(numberp (car lst))
				(numberp n)
			)
			(* (car lst) n)
		)
		(
			T 
			Nil
		)
	)
)
\end{lstlisting}

\end{document}