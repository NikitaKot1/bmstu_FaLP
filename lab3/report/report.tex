\documentclass[a4paper,14pt, unknownkeysallowed]{extreport}
\usepackage{placeins}
\include{preamble.inc}

\begin{document}

\include{00-title}

\section*{Задание 1}

\textbf{Написать функцию, которая принимает целое число и возвращает первое четное число, не меньшее аргумента.}

\begin{lstlisting}
(defun make_even (x)
	( 	
		if (evenp x)
		x
		(+ x 1)
	)
)
\end{lstlisting}

\section*{Задание 2}

\textbf{Написать функцию, которая принимает число и возвращает число того же знака, но с модулем на 1 больше модуля аргумента.}

\begin{lstlisting}
(defun abs_plus_one (x)
	(	
		if (< x 0)
		(- x 1)
		(+ x 1)
	)
)
\end{lstlisting}

\clearpage

\section*{Задание 3}

\textbf{Написать функцию, которая принимает два числа и возвращает список из этих чисел, расположенный по возрастанию.}

\begin{lstlisting}
(defun make_increase_list (a b)
	(
		if (< a b)
		(cons a (cons b Nil))
		(cons b (cons a Nil))
	)
)
\end{lstlisting}

\section*{Задание 4}

\textbf{Написать функцию, которая принимает три числа и возвращает Т только тогда, когда первое число расположено между вторым и третьим.}

\begin{lstlisting}
(defun is_first_between (first second third)
	(
		if 
			(
				or 
				(and (> first second) (< first third))
				(and (> first third) (< first second))
			)
		t
		nil
	)
)
\end{lstlisting}

\FloatBarrier
\section*{Задание 5}
\textbf{Каков результат вычисления следующих выражений?}

\begin{table}
	\begin{center}
		\begin{tabular}{|l|l|}
			\hline
			(and 'fee 'fie 'foe) & foe \\
			\hline
			(or nil 'fie 'foe) & fie \\
			\hline
			(and (equal 'abc 'abc) 'yes) & yes \\
			\hline
			(or 'fee 'fie 'foe) & fee \\
			\hline
			(and nil 'fie 'foe) & Nil \\
			\hline
			(or (equal 'abc 'abc) 'yes) & T \\
			\hline
		\end{tabular}
	\end{center}
\end{table}

\FloatBarrier

\section*{Задание 6}

\textbf{Написать предикат, который принимает два числа-аргумента и возвращает Т, если первое число не меньше второго.}

\begin{lstlisting}
(defun not_less (x y)
	(>= x y)
)
\end{lstlisting}

\section*{Задание 7}

\textbf{Какой из следующих двух вариантов предиката ошибочен и почему?}

\begin{enumerate}[label=\arabic*)]
	\item (defun pred1 (x) (and (numberp x) (plusp x)))
	\item (defun pred2 (x) (and (plusp x)(numberp x)))
\end{enumerate}

Вариант 2 ошибочен: в случае, если x не является числом, вызов функции plusp вызовет в программе ошибку.

\section*{Задание 8}
\textbf{Решить задачу 4, используя для ее решения конструкции: только IF, только COND, только AND/OR.}

~~

~~

~~

\textbf{IF:}

\begin{lstlisting}
(defun is_first_between (first second third)
	(
		if (> first second) 
		(
			if (< first third)
			t
			nil
		)
		
		(
			if (< first second) 
			(
				if (> first third)
				t
				nil
			)
		)
	)
)
\end{lstlisting}

\textbf{COND:}

\begin{lstlisting}
(defun is_first_between (first second third)
	(
		cond 
		(
			(> first second) 
			(
				cond ((< first third) t)
				     (T nil)
			)
		)
		
		(
			(< first second) 
			(
				cond ((> first third) t)
					 (T nil)
			)
		)
	))
\end{lstlisting}

\textbf{AND/OR:}

\begin{lstlisting}
(defun is_first_between (first second third)
	(
		or 
		(and (> first second) (< first third))
		(and (> first third) (< first second))
	)
)
\end{lstlisting}

\section*{Задание 9}

\textbf{Переписать функцию how-alike, приведенную в лекции и использующую COND, используя только конструкции IF, AND/OR.}

\textbf{IF:}

\begin{lstlisting}
(Defun how_alike (x y)
	(
		if (= x y) 'the_same
		(
			if (equal x y) 'the_same
			(
				if (oddp x)
				(if (oddp y) 'both_odd 'difference)
				(if (evenp y) 'both_even 'difference)
			)
		)
	) 
)
\end{lstlisting}


\textbf{AND/OR:}

\begin{lstlisting}
(Defun how_alike (x y)
	(
		or 
		(and (or (= x y) (equal x y)) 'the_same)
		(and (oddp x) (oddp y) 'both_odd)
		(and (evenp x) (evenp y) 'both_even)
		'difference 
	) 
)
\end{lstlisting}

\end{document}